\documentclass[12pt, letterpaper]{article}

% Loading packages
\usepackage{array, changepage, caption, cochineal, enumitem, float, geometry, graphicx, setspace, titlesec, xr-hyper}
\usepackage{hyperref}
\usepackage[utf8]{inputenc}
\usepackage[T1]{fontenc}
\usepackage[table]{xcolor}
\usepackage[hang, flushmargin]{footmisc}
\usepackage{tikz}

\usepackage[autocite=inline,
backend=biber,
doi=false,
style=apa,
natbib]{biblatex}

% Package options
\hypersetup{%
  colorlinks = true, 
  breaklinks = true, 
  urlcolor = black, 
  linkcolor = black,
  filecolor= black,
  citecolor = black,
} %%%
\urlstyle{same}

\addbibresource{references.bib}
\setlength\bibitemsep{0\itemsep}

\def\checkmark{\tikz\fill[scale=0.4](0,.35) -- (.25,0) -- (1,.7) -- (.25,.15) -- cycle;} 

\geometry{left = 1in, right = 1in, top = 1in, bottom = 1in}
\parindent = 25pt

\makeatletter
\newcommand*{\addFileDependency}[1]{\typeout{(#1)}
\@addtofilelist{#1}
\IfFileExists{#1}{}{\typeout{No file #1.}}
}\makeatother

\newcommand*{\myexternaldocument}[1]{%
\externaldocument{#1}%
\addFileDependency{#1.tex}%
\addFileDependency{#1.aux}%
}

\myexternaldocument{appendix}

\begin{document}

\titleformat*{\section}{\normalfont\bfseries}
\titlespacing*{\section}{0pt}{*4}{0pt}

\titleformat*{\subsection}{\normalfont\itshape}
\titlespacing*{\subsection}{0pt}{*4}{0pt}

\begin{titlepage}
    \begin{spacing}{1.3}
    \begin{center}
    \vspace*{0.5cm}

    \Large
    \textsc{Citizen Evaluations of Unequal Policy Representation \\
    A Pre-Analysis Plan
     } \\

    \vspace*{1cm}

    \normalsize
    \begin{tabular}{ccccc} 
    Mikael Persson & &  Wouter Schakel & & Christopher Wratil \\
    University of Gothenburg & & University of Amsterdam  & & University of Vienna \\
\\
    \end{tabular}

    \end{center}

    \vspace*{0.5cm}
    \begin{adjustwidth}{15pt}{15pt}
    \noindent


    \end{adjustwidth}
    \end{spacing}
\end{titlepage}

\begin{spacing}{1.5}


\section*{Research Questions}

Our main research question we aim to answer with the experiment is: Does inequality in substantive representation affect citizens’ satisfaction with democracy?

While we know that a lack of egotropic representation or outcome favorability diminishes political systems support (e.g. Mayne and Hakhverdian 2017; Anderson & Guillory, 1997; Anderson et al., 2005; Esaiasson et al., 2016, 2017; Ezrow and Xezonakis, 2011; Wratil and Wäckerle 2022), no work has addressed whether inequality in representation has an independent effect on citizens’ support. This is surprising given the large and growing literature on unequal representation in Western democracies (e.g. Gilens 2012; Gilens and Page 2014; Bartels 2016; Lupu and Warner 2021; Schakel 2019; Elsässer et al. 2020). The high scholarly interest in unequal representation, especially by income groups, arguably reflects a general concern that inequalities in representation could pose a challenge to democratic standards.

There exist different views of why and when inequalities in representation would be normatively challenging for democracy and our study tries to speak to several of them. First, if we consider inequalities in representation as normatively unjust per se, as they may contradict the idea of citizens as "political equals" (Dahl 1971), it is important to understand whether inequalities in representation lead to lower satisfaction with democracy. This is the case because if inequalities lower satisfaction with democracy, they may - in worst case - also destabilize democracy by undermining citizens' system support, or alternatively - in best case - nudge citizens to oppose unequal representation and reform democracy if increasingly dissatisfied with it. In turn, if inequalities in representation have little effect on satisfaction with democracy, this would suggest that citizens either do not care or not perceive inequalities in representation, which might mean that they will not engage in correcting them. In either case, knowing the effect of representational inequalities on citizens' satisfaction with democracy would help us understand the dangers hat unequal representation poses to democracy and the chances for making representation more equal.

Second, from a different perspective unequal representation may not always be normatively problematic per se but only to the extent that citizens disapprove of it. For instance, accounts that focus on the social legitimacy of representation as a yardstick to assess its normative quality suggest that if citizens approve of or - at least - accept inequalities in representation, we should pay provisional respect to their assessment as democratic agents (e.g. Saward 2010; Wolkenstein and Wratil 2021). Again, the effect of unequal representation on citizens' satisfaction with democracy is highly relevant. If citizens are less satisfied in face of inequalities in representation, correcting these inequalities becomes a normative imperative (as in the previous view). However, if citizens care little about inequalities in representation, they may be more normatively justifiable. From this perspective, it is a strong effect of unequal representation on satisfaction with democracy that mainly poses a normative challenge to democracy, whereas the absence of an effect may be more acceptable.

A third perspective might combine the different views and argue that while large inequalities in representation may always be normatively problematic, small inequalities may be normatively acceptable as long as they do not affect citizens' satisfaction with democracy. This perspective may be supported by the idea that “perfect” equality of democratic representation is illusive in modern democracies that rely on political institutions to ensure fair access to power and thereby inevitably and paradoxically create inequalities through these institutions (e.g. electoral systems that introduce thresholds and thereby differentially empower political forces with more resources). It may also resonate with contributions that argue that while representation may be unequal, its extent is unproblematic or negligible, for instance, as different groups often hold very similar preferences on most issues (e.g. Enns 2015; Branham et al. 2017). From this perspective (as from the previous one), unequal representation would primarily pose a normative challenge to democracy if it actually lowers satisfaction with democracy or other forms of citizens' support.

Given the high relevance of the link between unequal representation and citizens' views on democracy from various perspectives, we investigate whether unequal representation affects citizens' satisfaction with democracy - and to what extent. Our main hypothesis takes heed of scholars' concern that unequal representation violates the democratic ideal of equality and conjectures that citizens will also see unequal representation (ceteris paribus) as a negative cue about democratic quality. \\

\smallskip  \textit{Hypothesis H1: In general, citizens will express lower levels of satisfaction with democracy after being confronted with political outcomes of unequal as opposed to equal representation.}  \\

However, many citizens may just care about their own representation instead of unequal levels of representation, given limited sociotropic concerns. Hence, it is well conceivable that we find no or little support for H1 and our main aim is indeed to explore the relationship between unequal representation and satisfaction with democracy rather than test a well-established hypothesis.
We also consider several moderators of the relationship between unequal representation and citizens' satisfaction with democracy. First, we hypothesize that the effect of unequal representation may be dependent on whether citizens themselves are egotropically represented, which may diminish concerns for unequal representation, while being not represented themselves may increase such concerns. Essentially, we expect that the salience citizens attribute to equal representation varies by whether they get what they personally want and their personal representation makes them think less about the situation and potential inequalities, whereas a lack of egotropic representation activates equality concerns and feelings of relative deprivation compared to other groups.\\

\smallskip  \textit{Hypothesis H2: The effect of unequal representation on satisfaction with democracy will be weaker if a political outcome is in line with citizens' own policy preferences as opposed to being not in line with their policy preferences.}\\

A second potential moderator may be (perceived) membership in a group that is affected by unequal representation. If a group citizens belong to or identify with is opposed to a policy change, citizens may use this opposition as a trusted source cue that the policy has negative consequences for them to substitute for a lack of information about the policy (e.g., Lupia 1994; Lupia and McCubbins 1998). Citizens regularly form opinions about policies on the basis of group endorsements (Arceneaux and Kolodny 2009; Nicholson 2012). Hence, realizing that the group they identify with supports a decision may divert their attention away from the question of equality in representation and thereby diminish its effect on satisfaction with democracy.\\

\smallskip  \textit{Hypothesis H3: The effect of unequal representation on satisfaction with democracy will be weaker if a political outcome is in line with the policy preferences of the group citizens belong to as opposed to being not in line with this group's policy preferences.}\\

Third, we expect that perceptions of the deservingness of groups that are better/worse represented by a political outcome could influence how much inequality in representation weighs on satisfaction with democracy. If citizens perceive a group to be particularly deserving (e.g. in terms of deserving to be heard and helped by society), they will be more sensitive to a situation where this group is underrepresented compared to other groups.\\

\smallskip  \textit{Hypothesis H4: The effect of unequal representation on satisfaction with democracy will be weaker (stronger) if a political outcome is in line with the policy preferences of a group citizens consider to be deserving (undeserving) as opposed to being not in line with this group's policy preferences.}\\

Fourth, we expect that citizens' support for the principles of political and economic equality could influence the relationship between unequal representation and satsifaction with democracy. Support for the value of equality varies at the individual level and those with stronger equality preferences might view unequal representation as more problematic for society.\\

\smallskip  \textit{Hypothesis H5: The more citizens value political and economic equality as societal principles, the stronger will be the effect of unequal representation on satisfaction with democracy.}\\

Finally, we also conjecture that feelings of relative deprivation could moderate the effect between unequal representation and satisfaction with democracy. Studies have shown that relative deprivation is associated with negative throught processes and attitudes towards the political system, such as populist attitudes (e.g. Elchardus and Spruyt 2016). We therefore expect that citizens who feel that they do not get as much from society and politics as other groups could react more negatively to unequal representation in general, and especially if they themselves are not represented.\\

\smallskip  \textit{Hypothesis H6: The stronger citizens' feelings of relative deprivation are, the stronger will be the effect of unequal representation on satisfaction with democracy. This moderation effect will be especially pronounced if a political outcome is not in line with citizens' own policy preferences.}\\

Note that it is also conceivable that in case a political outcome is in line with citizens' personal preferences individuals with strong feelings of relative deprivation may even prefer unequal over equal representation in order to "make up" or "take revenge" for previous deprivation vis-à-vis other groups that they perceive. In this case, the overall effect of relative deprivation on the relationship between unequal representation and satisfaction with democracy may even be zero, while being moderated in opposite directions by egotropic representation.

In addition to these hypotheses, which we test through a multifactorial vignette survey experiment, we can also test variations of them (except for Hypothesis H2) through a framing experiment, in which we present citizens with different frames of which research findings on unequal representation could be emphasized. These frames either emphasize that the rich are better represented or that there is equality in representation. This leads to more specific expectations for the variations of Hypotheses H3 and H4. We list all variations below:\\

Hypothesis H1 (Framing Experiment): In general, citizens will express lower levels of satisfaction with democracy after being confronted with a presentation of research findings that emphasizes inequalities in policy representation in their country as opposed to emphasizing equalities.\\

Hypothesis H3 (Framing Experiment): The framed effect of unequal representation on satisfaction with democracy will be weaker for high- than low-income citizens.\\

Hypothesis H4 (Framing Experiment): The framed effect of unequal representation on satisfaction with democracy will be weaker if citizens consider the rich to be deserving and the poor to be undeserving (and vice versa).\\

Hypothesis H5 (Framing Experiment): The more citizens value political and economic equality as societal principles, the stronger will be the framed effect of unequal representation on satisfaction with democracy.\\

Hypothesis H6 (Framing Experiment): The stronger citizens' feelings of relative deprivation are, the stronger will be the framed effect of unequal representation on satisfaction with democracy.\\

Note that Hypothesis H2 and the second part of Hypothesis H6 are not testable in the framing experiment, as it does not show a single political outcome on which we could elicit respondents' preferences (see the next sections).\\

\section*{Study design}

This study consists of two online survey experiments: a multifactorial vignette experiment and an a framing experiment with two emphasis frames.

The multifactorial vignette experiment will ask respondents to assess five different situations, in which the national government either decides against or in favor of a specific policy proposal. In addition to the government's decision, respondents receive stimuli about the level of public support for the policy proposal in three equally-sized income groups (low-, medium-, and high-income people). These stimuli determine for each government decision whether the groups are equally or unequally represented and to what extent. We also provide the overall support for the policy proposal across income groups, as respondents may otherwise try to infer it from the figures of support in the individual groups and, for instance, erroneously assume that the medium-income group is the biggest group and therefore deduce overall support from their support.

Moreover, we also include the position of organized interest groups on the policy as well as the popularity of the government as further attributes in the vignettes. First, we are concerned that respondents could try to infer the positions of organized interest groups from the vignettes if we do not provide their position directly. For instance, if public support for a proposal was high, only the high-income group was opposed, but the government decides against it, respondents may assume that it follows interest groups with similar preferences to high-income citizens and this could very negatively affect their satsifaction of democracy. This would potentially bias our estimates of the effect of unequal representation on citizens' satisfaction with democracy, because part of the estimated effect would rather reflect that people do not like a government that follows interest groups. Similarly, if a government takes a decision in line with all income groups' preferences, respondents may infer that the government is popular and this may also increase their satisfaction with democracy, alongside any direct effect of equal representation. To isolate the effect of unequal representation, we therefore vary the popularity of the government, for instance, including scenarios in which an unpopular government realizes equal representation.

Before respondents see the first vignette, they get the following debriefing text to avoid deception:

"In the following, we will show you figures of public support for five proposed policy changes and ask you to what extent decisions on these proposals would affect your satisfaction with democracy in the [US / Sweden]. While we have made up the survey results and government decisions, it is possible that such proposals are advanced in the future and that the government and public would react in the displayed ways. Hence, we would like to know how you would react to such a situation. Please note that we will show you survey results for low-, medium- and high-income people. Each of these groups is made up of a third of the population."

Each vignette then starts with the the following sentence:

"Some people want to POLICY PROPOSAL. Others want to leave things as they currently are.
A recent survey revealed the following views on the issue among different income groups in [the US / Sweden]:"

Here, the POLICY PROPOSAL comes from a set of 24 policy issues that could be decided by the national government in each country and that we selected on basis of their coverage in public opinion polls in the US and Sweden. Specifically, we first searched for questions on policy proposals that had been asked in 2023 and 2022 by major polling organizations in each country (e.g. Gallup in the US), creating a long list of "candidate" policy proposals. Note that using survey coverage of policy proposals as the basis to define the set of issues closely mimicks how the most eminent studies on unequal representation selected their policy issues (e.g. see Gilens 2012). Based on the long list of candidate proposals from each country, we then purposively chose proposals that fulfilled the following criteria. First, for each country, we selected 12 policy issues that are country-specific and make little sense in the other country context (e.g. merging Sweden's four electricity areas) and 12 "general" proposals that could be relevant in both countries (e.g. increasing military aid to Ukraine). This accounts for the different political agendas of the two countries while ensuring direct comparability between the cases. Moreover, for each set of 12 issues, we selected six policy proposals with redistributive consequences and six with no obvious redistributive relevance to be able to ascertain whether unequal representation on policies that have stronger implications for different income groups impact more heavily on satisfaction with democracy. Finally, we also aimed for selecting policy proposals that vary in the extent to which we would expect citizens' preferences to correlate by their income, ensuring that we also include issues on which it is plausible that different groups could take all kind of preferences. The full list of policy proposals with their classification as country- specific or applicable in both countries as well as redistributive vs. non-redistributive is attached to this pre-analysis plan. In general, using our high number of 36 policy proposals across two countries (12 per country, plus 12 shared) also ensures that any overarching findings will not be due to idiosyncratic effects of a small number of specific issues (e.g. see Blumenau and Lauderdale 2023). Upon entering the survey, we randomly draw five out of the 24 policy issues (without replacement) for each respondent - one for each vignette the respondent sees.

After the issue is introduced, the vignette presents a table with two columns containing the figures of public support for the policy 
proposal in each income- group. The income groups are listed in the left column. We randomly draw at the respondent level whether the low-income group's or the high-income group's preferences appear at the top of the table; the medium-income group's preferences always appear in the middle. Hence, the resulting order from top to bottom is either "low-income people", "medium-income people", and "high-income people", or "high- income people", "medium-income people", and "low-income people". The left column displays in a sentence the percentage (between 1-99\%) of low-/medium- /high-income people that supported the proposal in the survey.
Here, we implement two experimental arms, in which respondents are randomly assigned when entering the survey: a "hypothetical results" arm and a "realistic results" arm. In the hypothetical results arm, we draw the survey results for each income group independently from a standard normal distribution with a mean of 50 and a standard deviation of 23. We round the drawn value to full numbers. We chose the standard deviation of 23 as this number is very close to the standard deviation of 21 in the datasets used by Gilens (2012) and Persson (2023) - for the US and Sweden respectively - and ensures that only very few drawn values fall outside 1(\%) and 99(\%), which we view as plausible boundaries of support. Whenever our drawn value falls outside the 1-99\% boundaries we take another draw from the distribution until the value lies inside. Note that this slighty decreases the standard deviation of the drawn treatment values, which will be below 23 and close to 21 (like in Gilens 2012 and Persson 2023). We also discard all draws of 50, since this value does not clearly convey whether a majority or minority supports the proposal. Again, we simply draw again from the same distribution. The big advantage of the hypothetical results arm is that the different group's preferences are not correlated at all. This creates a lot of vignettes in which unequal representation occurs. For instance, in expectations only in 25\% of the draws will all income groups have the same majority preference on the policy proposal (i.e., all in favor or all against), leading to equal representation (all represented or all not represented). The 75\% remaining cases represent different forms of unequal representation.

The disadvantage of this experimental arm is that it does not account for the fact that groups' preferences are highly correlated in reality (e.g. see Bashir 2015), fuelling doubts about the ecological validity of this design (e.g. as respondents may see many scenarios they may perceive as implausible). The realistic arm addresses this problem. Here, we do not draw groups' preferences independently from each other but we impose correlations between the groups' preferences that most closely resemble those present in Gilens' (2012) data, who reports a correlation of 0.78 between the high- and medium-income groups. We aim for the same correlation for the high-medium and the low-medium correlations. To achieve this, we define the correlations slightly higher than 0.78 (0.82) due to our rejection of draws outside the 1-99\% boundaries and use a Cholesky decomposition of the correlation matrix, which we then multiply with three random draws from the normal distribution (for each of the income groups). As a result, we obtain values where the correlations between the high- and medium-income groups' preferences are approximately 0.78 and between the high- and low-income groups' preferences approximately 0.67. The percentage of vignettes in which the majority preferences of the three groups are diverging falls to 33\%. Note that the realistic results arm does not address the issue that the level or direction of support may still be unrealistic (e.g. a proposal is supported by all groups that would have little support in reality or it is most supported by a group that is likely to support it least in reality). It only ensures that the preferences of the different income groups are to some extent similar on average and that there is a tendency for medium- and low- as well as medium- and high-income groups' preferences to be more similar than low- vs. high-income groups' preferences. We attach the Javascripts (for the US and the Swedish survey) implementing the two treatment arms to this pre-analysis plan.
Below the table we provide treatments for four attributes. First, we say "The average support across all income groups for POLICY PROPOSAL was AVERAGE SUPPORT\%", where AVERAGE SUPPORT is the simple rounded average of the support across the three income groups (see above). Second, we state "Organized interest groups are mostly INTEREST GROUP POSITION the proposed policy", which randomly takes the values of "supportive of", "critical of", "neutral towards" or "divided over". Third, we provide a cue towards government popularity by saying "The government is generally GOVERNMENT POPULARITY among citizens", using "very popular", "rather popular", "rather unpopular", and "very unpopular" as the attribute levels. Finally, we state the outcome, i.e. the government's decision: "In the end, the government ADOPTION POLICY PROPOSAL." Here, the attribute levels are "decides to go ahead with" and "decides against" and the policy proposal is repeated in aning form.

Each situation they assess on two items. The first item serves as a manipulation check of whether we could influence their perceptions of (un)equal representation with the vignettes. Specifically, we ask: "Do you think that one or two groups get more of what they want from this decision than the other(s)?" Respondents can tick multiple answers from the following list: "Yes, the low-income people", "Yes, the medium-income people", "Yes, the high-income people", "No", and "Don't know" (but not combine the "Yes" answers with "No" or "Don't know"). The second item measures our key dependent variable, satisfaction with democracy, on a five-point scale. To avoid that any results are due to a specific formulation of the dependent variable, we randomize on the respondent level between four different wordings of the question including varying answer options:\\

WORDING 1: "To what extent does this decision increase or decrease your support for the way democracy works?", with answer options "Very much increase", "Increase", "Neither increase nor decrease", "Decrease", and "Very much decrease".\\
WORDING 2: "Would this decision make you more or less satisfied with the way democracy works?", with answer options "Much more satisfied", "More satisfied", "Does not affect my satisfaction", "Less satisfied", and "Much less satisfied".\\
WORDING 3: "How would this decision affect your satisfaction with how democracy works?", with answer options "Very positively", "Positively", "Neither positively nor negatively", "Negatively", and "Very negatively".\\
WORDING 4: "Thinking about this decision, do you feel more or less satisfied with the way democracy works?", with answer options "Much more satisfied", "More satisfied", "Does not affect my satisfaction", "Less satisfied", "Much less satisfied".\\
After having completed all five tasks for the vignette experiment, all respondents participate in the framing experiment. The introductory text is shown to all respondents:

"Though the scenarios you have just evaluated were made up, political scientists have analyzed real proposals, survey results and government decisions in [the United States / Sweden]. For this, they have used many hundreds of policy changes across several decades."
Below this text, respondents randomly see one of two emphasis frames. The basic idea behind the emphasis frames is to exploit the fact that most studies on unequal representation find that policy is largely responsive to the high-income rather than medium- or low-income groups, whereas policy congruence with majority opinion is usually similar across income groups. This paradoxical result has been reported for the US as well as Sweden (e.g. Gilens 2012; Persson 2023). Hence, by showing respondents one of these frames, we are indeed only emphasizing one perspective on the (in)equality of policy representation, avoiding any form of deception about research results. The two frames are as follows:\

FRAME RESPONSIVENESS: "Some of this research has found that federal government policies in [the US / Sweden] tend to be influenced mostly by high- income citizens. In fact, it matters little what the majority of low-income and middle- income citizens want government to do; decisions rarely go their way, unless high- income citizens want the same thing. As a result, the high incomes have a much greater say over government policy than do the lower incomes."

FRAME CONGRUENCE: "Some of this research has found that fedral government policies in [the US / Sweden] tend to reflect the views of all income groups to a similar extent. On some issues, the majority of high-income citizens get the decision they want; on other issues, decisions align more with the preferences of low-income and middle-income citizens. As a result, the match between what policies people want and which policies they actually get is similar for groups of citizens with different incomes."
Below the frame, we measure the dependent variable in a very similar form as in the multifactorial vignette experiment using the following four wordings (note that respondents who receive "WORDING X" on the first experiment, also receive "WORDING X" on the second experiment):
WORDING 1: "To what extent does this research finding increase or decrease your support for the way democracy works?", with answer options "Very much increase", "Increase", "Neither increase nor decrease", "Decrease", and "Very much decrease".
WORDING 2: "Does this research finding make you more or less satisfied with the way democracy works?", with answer options "Much more satisfied", "More satisfied", "Does not affect my satisfaction", "Less satisfied", and "Much less satisfied".
WORDING 3: "How does this research finding affect your satisfaction with how democracy works?", with answer options "Very positively", "Positively", "Neither positively nor negatively", "Negatively", and "Very negatively".
WORDING 4: "Thinking about this research finding, do you feel more or less satisfied with the way democracy works?", with answer options "Much more satisfied", "More satisfied", "Does not affect my satisfaction", "Less satisfied", "Much less satisfied".

In order to test hypotheses H2 to H6 and H3 (Framing Experiment) to H6 (Framing Experiment) on the moderation of the effects of unequal representation, we also measure the respective moderators before the first experiment. For hypothesis H2 we must capture whether a political decision in the vignettes is in line with respondents' personal preferences. To this end, we ask respondents: "Below, we list several policy proposals that could be implemented by the federal government in the future. Please indicate to what extent you personally favor or oppose each proposal." The wordings of all proposals that are later shown to respondents in the vignette experiment are listed and each proposal is rated using the answer categories "Strongly oppose", "Somewhat oppose", "Somewhat support", and "Strongly support".
For hypothesis H3 we must measure respondents' belonging to the different income groups. For this purpose we use two questions: one on reported and one on perceived income. The first question reads as: "In which of the following categories did your total family income, before taxes, fall last year? Total family income includes money from jobs, net income from business, farm or rent, pensions, dividends, interest, social security payments and any other money income received. If you don't know the exact figure, please give an estimate." The exact answer options are available in the full questionnaires that are attached to this pre-analysis plan. The second question is: "Compared with [American / Swedish] families in general, would you say your family income is far below average, below average, average, above average, or far above average? If you are uncertain, just give us your best guess." The answer options are "Below average", "Average", "Above average", "Far above average", and "Don't know".

For hypothesis H4 we measure to what extent respondents view high- and low- income groups as "deserving" (we omit attitudes towards medium-income groups). Specifically, we use a five-point agree-disagree Likert-item battery with the following four statements: 1) "Society should take care of the poor, regardless of what they give back to society." 2) "Society should take into account the perspectives of the rich, regardless of how much they contribute to society." 3) "The poor deserve to have their views taken seriously in politics." 4) "The rich deserve to have their views taken seriously in politics."

For hypothesis H5 we must operationalize how much citizens value political and economic equality as societal principles. For this, we use two questions to be answered on a seven-point scale. The first concerns political equality: "Some people think that everyone should have about an equal say in government policy. Others think that some people should have more influence than others, for instance, because they know more about policy or they contribute more in taxes. What is your own view on this?" We only label the ends of the scale with "1 - Everyone should have an equal say" and "7 - Some people should have more influence than others". The second question concerns economic eqaulity: "Some people think that differences in income should be minimal to ensure a decent living standard for everyone. Others think that there should be large differences in income, for instance, because some people have more talent or put in more effort than others. What is your own view on this?" The labelled scale ends are "1 - Income differences should be minimal" and "7 - Income differences should be large".

For hypothesis H6 we have to measure feelings of relative deprivation. We rely on a shortened item battery that was initially proposed by Elchardus and Spruyt (2016). While these authors rely on seven items, we use a shorter five-point agree-disagree Likert-item battery composed of the four items that attained the highest factor loadings in the confirmatory factor analysis these authors performed. They are: 1) "The government doesn't do enough for people like me, others are always advantaged." 2) "Whichever way you look at it, we are the kind of people that never get a break." 3) "When we need something from the government, people like us always have to wait longer than most." 4) "It is always other people who can profit from all kinds of advantages offered in this society."
Finally, we measure a host of other "standard" socio-demographic and political covariates ranging from gender, ethnicity and occupation to partisanship, liberal- conservative/left-right preferences and (pre-treatment) satisfaction with democracy. The full list of covariates can be checked in the full questionnaires that are attached to this pre-analysis plan.



\subsection*{Data collection procedures}


The survey will be conducted in nationally representative samples of the voting- eligible population (in federal/national elections) with residency in the USA and Sweden. The target sample sizes for each country are n = 3,000. The surveys will be conducted online in the survey software Qualtrics. Participants will be sourced through the Cint marketplace platform and incentives for participants vary by the specific providers/panels that Cint uses. In order to ensure national representativeness, we employ quotas on age * gender, region and education in both countries based on (micro-)census information. The exact quotas for both countries are attached to this pre-analysis plan. Best efforts will be made to fill all quotas, in particular, by extending the fieldwork period. However, we expect that some quotas (especially on education) have to be collapsed or relaxed towards the end of data collection. Depending on the extent of this issue, we may decide to use post-stratification weights in particular analyses.

Upon providing their consent to participation, respondents will have to pass a very basic attention check that asks them who is conducting the survey - an information that is highlighted in the consent form they just agreed to. We provide "Researchers from universities" as the correct answer as well as "A large corporation" and "The [Swedish / US] government". We will deny access to the survey to those participants that do not pass this basic attention check to avoid lowest-quality responses (e.g. straightlining). This measure is also helpful to screen out respondents who are potentially mentally impaired and should not participate in social science research. We also terminate all respondents that do not fit (anymore) the national quotas and ensure that all respondents are at least 18 years old, are resident in the respective country and have national citizenship. Finally, we exclude from the sample all respondents who complete the entire survey in less than three minutes, as we estimate the average response time to be about 8-10 minutes. Accordingly, such respondents may not have paid attention at all to the survey. While not counting these respondents towards the quotas, we preserve their data, which allows us to assess any impacts that the inclusion of "speeder" respondents may have on our estimates at the analysis stage.
Our aim is to start data collection in the week 14-20 August 2023 and complete most of the collection by the end of August. We will however consider extending fieldwork throughout September and October 2023 if this is necessary to fill all quotas. We do not expect any period effects on any of our key estimates of interest. A long fieldwork period is thus unproblematic from our perspective and potentially desirable if it increases sample quality.





\subsection*{Variables}

For both experiments, our dependent variable is satisfaction with democracy measured on a five-point scale with four different wordings per experiment. We will pool data from all wordings and use a variable coded 1-5 (lowest to highest expected change in satisfaction) as the dependent variable in all analyses for each experiment. We will treat this variable as continuous.

In the multifactorial vignette experiment, the main independent variable will be inequality in representation. This variable can be operationalized in a number of different ways. As baseline we will operationalize it as a dummy variable that is "0" if there is majority agreement on the proposal, i.e. if public support for the proposal shown in the vignette is below or above 50\% for all income groups. If the majority preferences of all income groups concur, their representation will be equal, irrespective of whether the proposal is implemented or not. In turn, the dummy variable is "1" if the displayed public support is below 50\% in at least one of the groups, while it is above 50\% in at least one other group.

This baseline measure of unequal representation focuses on differences in the representation of the majority in each group. It does not account for more fine- grained degrees of unequal representation, which much of the observational literature focuses on. We therefore complement it with alternative operationalizations. First, we use a more continuous measure of unequal representation that is the distance in displayed public support between the income group that supported a policy proposal most and the income group that supported it least. The idea here is that situations in which the support is 80\%, 80\%, 20\% vs. 20\%, 20\%, 80\% vs. 80\%, 50\%, 20\% all have the same degree of inequality in representation, irrespective of whether the proposal is implemented or not. However, these situations are more unequal than – say – situations with 40\%, 50\%, 60\% support. Compared to the baseline measure, the majority preferences in each group are irrelevant for this measure.

Second, we also operationalize three relative measures of unequal representation that compare the extent of representation between two of the groups respectively. 1) Unequal representation high vs. low is the percentage of people in the high- income group who got their preferences realized through the political outcome minus the percentage of people in the low-income group who got their preferences realized. 2) Unequal representation high vs. medium is the percentage of people in the high-income group who got their preferences realized through the political outcome minus the percentage of people in the medium-income group who got their preferences realized. 3) Unequal representation medium vs. low is the percentage of people in the medium-income group who got their preferences realized through the political outcome minus the percentage of people in the low- income group who got their preferences realized. While we have not theorized this conjecture, these measures allow us to assess whether unequal representation in favor of the higher income groups specifically affects satisfaction with democracy.

The dependent and independent variables allow us to test our major hypothesis H1. In order to test H2, we operationalize a dummy variable that is "1" if the respondent's personal preference on an issue is in line with the political outcome shown in the vignette and "0" otherwise. We measure respondents' preferences on all shown proposals pre-treatment (see "Design Plan"). The dummy is coded as "1" if the proposal is implemented and the respondent either ticked "Somewhat support" or "Strongly support" in the pre-treatment question as well as when it is not implemented and the respondent selected one of the "oppose" options. It is "0" otherwise. "Don't know" responses are treated as missing values on this variable.

To test H3, we use two dummy variables for whether a political outcome is in line with the preferences of a group respondents belong to, based on our questions on reported and perceived income (see "Design Plan"). For reported income, we treat respondents with an income below the 33rd percentile to belong to the low-income group, those with an income between the 33rd and below 67th percentile to belong to the medium-income group, and those on the 67th percentile or higher to belong to the high-income group. For perceived income, we treat respondents that select "Far below average" or "Below average" to belong to the low-income group, "Average" to belong to the medium-income, and "Above average" as well as "Far above average" to belong to the high-income group. Both dummy variables are "1" if the majority preferences of the group the respondent belongs to are realized in the political outcome shown in the vignette (i.e., public support in the group is over 50\% and the proposal is implemented, or it is below 50\% and the proposal is not implemented) and "0" otherwise. For both variables, "Don't know" responses on the reported or perceived income questions lead to a missing value on the dummy variables.

To test H4, we operationalize two variables for the deservingness of the rich and the poor as the simple averages of the answers to the two agree-disagree items on the deservingness of the rich and the poor respectively, where we use values from 1 "Strongly disagree" to 5 "Strongly agree". We treat these variables as continuous in the analysis models. For both variables, "Don't know" responses on any of the related deservingness perceptions items lead to the coding of the dummy variables as missing value. At the analysis stage, the variables are interacted with a dummy variable on whether the majority preferences of the rich (poor) are realized in the political outcome shown in the vignette. Note already that this leads to three-way interactions between unequal representation, deservingness perceptions and representation of the (un)deserving groups to test this hypothesis, while most other hypotheses are tested with two-way interactions.
To test H5, we operationalize two variables as the simple responses on respondents' valuations of political and economic equality questions (see "Design Plan"), creating variables that range from "1" to "7", with "Don't know" responses treated as missing values. We treat these variables as continuous in the analysis models. To test H6, we operationalize feelings of relative deprivation as the mean response to the items in the relative deprivation battery (see "Design Plan"), coding responses from "1" ("Strongly disagree") to "5" ("Strongly agree"). If more than two items have been answered with "Don't know" by a respondent, we code a missing value for the variable; if only one or two items have a "Don't know" response, we still calculate the mean for the remaining items. At the analysis stage, the variable is interacted with a dummy variable on whether the respondent's personal preference is realized in the political outcome shown in the vignette (see the test for H2 above). Note already that this leads to a three-way interaction between unequal representation, feelings of relative deprivation and representation of the respondent's personal preference.

As unequal representation is orthogonal to the other attributes in the vignette experiment, variables derived from them are not strictly necessary to test hypotheses H1 to H6. However, as it is customary in multifactorial vignette experiments, we still operationalize variables relating to these attributes to gauge their effects on the outcome: 1) We operationalize three continuous variables capturing the public support in each income group respectively as well as public support across groups (ranging from 1-99). 2) For the position of organized interest groups, the popularity of the government and the adoption of the policy proposal, we operationalize categorical variables containing all the relevant levels (see "Design Plan").

To assess experimental manipulation, we will also create a dummy variable that is "1" if respondents perceive at least one group to get more of what they want from a decision and "0" if they answer "No" instead (see "Design Plan" on the manipulation check question). "Don't know" answers are treated as missing values on this dummy variable.

For the framing experiment, our main independent variable will be a dummy variable for unequal representation that is "1" if the responsiveness frame is shown and "0" if the congruence frame is shown to respondents. This is used to test H1 (Framing Experiment). To test H3 (Framing Experiment), we will use two categorical variables that assign respondents into low-, medium- and high-income categories. One will do so for reported income and one for perceived income and the operationalizations of the categories are identical to those used to test H3 (see above). To test H4 (Framing Experiment) and H5 (Framing Experiment), we will use the same variables operationalizing the (un)deservingness of the rich and the poor as well as respondents' valuations of political and economic equality as those used to test H4 and H5, respectively (see above). To test H6 (Framing Experiment), we will also rely on the same variable operationalizing feelings of relative deprivation as used to test H6, but we cannot interact it with whether the respondent's preference is realized in the outcome as is possible in the vignette experiment.

Finally, our survey also measures a host of other factors that are not essential to test our key hypotheses but that may serve during the development of this research to refine our understanding of results and probe their robustness (e.g. attention of respondents, partisanship, pre-treatment satisfaction with democracy). As any analyses employing these factors will necessarily be exploratory, we do not specify ex ante how we would operationalize related variables.


\section*{Analysis Plan}

We will test our hypotheses using linear regression models. The analyses will be carried out for each of the two countries separately.
In the multifactorial vignette experiment, to test H1 we will regress our measure of satisfaction with democracy on our dummy variable for unequal representation as well as the variables related to the other attributes in the vignettes (see "Variables"). To test H2, we will add an interaction effect between unequal representation and the dummy variable on whether the respondent's personal preference is realized in the political outcome. To test H3, we will add an interaction effect between unequal representation and the dummy variable on whether the majority preferences of the group the respondent belongs to are realized in the political outcome, running separate models for determining group membership on the basis of reported vs. perceived income. To test H4, we will add two three-way interactions between unequal representation, the (un)deservingness perceptions of the rich and the poor (respectively) as well as dummy variables on whether the majority preferences of the rich/poor are realized in the political outcome. To test H5, we will add two interaction effects between unequal representation and respondents' valuations of political vs. economic equality respectively. To test H6, we will add a three-way interaction effect between unequal representation, respondents' feelings of relative deprivation and the dummy variable on whether the respondent's personal preference is realized in the political outcome. Note that here we assess the significance of one of the constitutive two-way as well as the three-way interaction, i.e. a moderation by feelings of deprivation and a moderation by feelings of deprivation specifically if own policy preferences are not realized. In all cases, all constitutive terms of the two- and three-way interactions will also be added to the models. Each hypothesis will be tested with a separate model and H3 with two separate models for the different income measures, resulting in seven main models using the baseline operationalization of the unequal representation variable. Moreover, we will rerun all these models with the alternative operationalizations of unequal representation outlined in the "Variables" section. All models will be implemented with clustered standard errors by respondent as well as (as a robustness check) by policy issue of the vignette.

To check experimental manipulation, we will run mixed-effects logistic regression models regressing our dummy variable of whether at least one group gets more of what they want from the decision on the different unequal representation measures. These models will only include the independent variable as well as a random intercept for respondents. If we find that manipulation is weak, we will consider to re-estimate models using a two-stage setup to identify the effect of unequal representation on satisfaction of democracy for complying respondents only.

We will also perform some split-sample analyses by experimental arm. In particular, we expect that citizens may be used to the small differences in representation displayed in the "realistic results" arm and that the effects of unequal representation could be weaker in this arm. However, at the same time, the public support figures in the "hypothetical results" arm may appear exaggerated to citizens and they may comply less with the experimental treatments as a consequence. Hence, while we have some expectations, we mainly view these analyses as exploratory.

In the framing experiment, to test H1 (Framing Experiment) we will regress satisfaction with democracy on the dummy variable for unequal representation. To test H3 (Framing Experiment), we will add an interaction effect between unequal representation and the categorical variable indicating a respondent's income group (note, this results in two interaction terms, for two of the three income categories, being added), running separate models for reported vs. perceived income. To test H4 (Framing Experiment), we will add two interactions between unequal representation and the (un)deservingness perceptions of the rich as well as the poor (respectively). To test H5 (Framing Experiment), we will add two interaction effects between unequal representation and respondents' valuations of political vs. economic equality respectively. To test H6 (Framing Experiment), we will add an interaction effect between unequal representation and respondents' feelings of relative deprivation. As in the multifactorial vignette experiment, we will add all constitutive terms of the interactions to the models and test each hypothesis in a separate model.

We will also assess the robustness of our main findings: 1) For the multifactorial vignette experiment, we will assess the variation in the effect of unequal representation across the different policy issues using mixed-effects linear regression models with random intercepts and slopes at the policy issue level. We will also rerun the main models using only redistributive vs. non-redistributive as well as country-specific vs. general issues. 2) For both experiments, we will investigate whether the effect of unequal representation varies by the attention that respondents paid to the survey using a single attention check implemented in the relative deprivation battery, where we ask respondents to click "Strongly agree" to demonstrate their attention as a fifth item. 3) For both experiments, we will also rerun the main models for each of the four wordings of the dependent variable question to assess whether results differ starkly depending on the wording used. Note that this test may lack power for the framing experiment.
To assess whether the relative deprivation battery is applicable in our samples, we will run exploratory factor analyses on the four items in each country. We expect absolute factor loadings above 0.45 for each item with a single factor being retained according to the Kaiser-Guttman criterion of an eigenvalue >1. If these conditions are not met, we will reconsider whether and how we can operationalize feelings of relative deprivation in our sample.




\end{spacing}

\vspace*{-.5cm}
\begin{spacing}{1.5}
\printbibliography
\end{spacing}

\end{document}